% Naming convention:
%    <above/below : a/b><level: #>_<#cm from origin>
%    Examples:
%       a2_0 = above, level 2, at origin
%       b3_1 = below, level 3, 1cm from origin

\begin{tikzpicture}
  \begin{feynman}
    
    % START: Horizon ---------------------------------
    \vertex (horizon);
    \vertex[right=5cm of horizon] (hz_5);
    \vertex[right=7cm of horizon] (hz_7);
    % END: Horizon -----------------------------------
    
    % START: Above -----------------------------------
    % Level +2
    \vertex[above=2cm of horizon] (a2);
    \vertex[left=1cm of a2] (1_a2) {\(g\)};
    \vertex[right=2cm of a2] (a2_2);
    \vertex[right=11cm of a2] (a2_11) {\(u,s\)};
    % END: Above -------------------------------------
    
    % START: Below -----------------------------------
    % Level -1
    \vertex[below=1cm of horizon] (b1);
    \vertex[right=11cm of b1] (b1_11) {\(\overline u,\overline s\)};
    % Level -2
    \vertex[below=2cm of horizon] (b2);
    \vertex[left=1cm of b2] (1_b2) {\(g\)};
    \vertex[right=2cm of b2] (b2_2);
    \vertex[right=9cm of b2] (b2_9);
    % Level -3
    \vertex[below=3cm of horizon] (b3);
    \vertex[right=11cm of b3] (b3_11) {\(\gamma\)};    

    \diagram* {
      {[edges=gluon]
        (1_a2) -- (a2_2),
        (1_b2) -- (b2_2),
      },
      
      (a2_2) -- [fermion, edge label=\(t\)] (hz_5),
      (b2_2) -- [anti fermion, edge label=\(\overline t\)] (hz_5),
      (a2_2) -- [anti fermion, edge label=\(\overline t\)] (b2_2),
      
      (hz_5) -- [scalar, edge label=\(H\)] (hz_7),
      
      (hz_7) -- [fermion] (a2_11),
      (hz_7) -- [anti fermion, edge label=\(\overline u\,\overline s\)] (b2_9),
      
      (b2_9) -- [anti fermion] (b1_11),
      (b2_9) -- [photon] (b3_11),
    };
    
    \draw [decoration={brace}, decorate] (a2_11.north east) -- (b1_11.south east)
          node [pos=0.5, right] {\(\rho, \phi\)};
  \end{feynman}
\end{tikzpicture}