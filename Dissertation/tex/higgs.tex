\begin{section}{Motivation}

\cite{Maldacena:1997re,joesbook}. Figure \ref{fig:label}.

\end{section}

\begin{section}{Analysis Methods}
\begin{subsection}{Data}
The analysis begins with data based on a sample of proton-proton collisions collected by the Compact Muon Solenoid (CMS) detector in the LHC. ``Interesting'' events are selected by the first level of the CMS trigger system which uses information from the detector's calorimeters and muon detectors to select events for analysis in a fixed time interval of less than 4 $\mu s$. These events are then further processed by a high-level trigger processor farm, which decreases the event rate from around 100 kHz to less than 1 kHz, before the data is stored. Finally, the particle-flow algorithm reconstructs and identifies all particles from the events selected by the CMS trigger system. With the data properly processed and promptly reconstructed ``online,'' further analysis can be carried out ``offline.''
\end{subsection}
\begin{subsection}{Monte Carlo}
The analysis begins with data based on a sample of proton-proton collisions collected by the Compact Muon Solenoid (CMS) detector in the LHC. ``Interesting'' events are selected by the first level of the CMS trigger system which uses information from the detector's calorimeters and muon detectors to select events for analysis in a fixed time interval of less than 4 $\mu s$. These events are then further processed by a high-level trigger processor farm, which decreases the event rate from around 100 kHz to less than 1 kHz, before the data is stored. Finally, the particle-flow algorithm reconstructs and identifies all particles from the events selected by the CMS trigger system. With the data properly processed and promptly reconstructed ``online,'' further analysis can be carried out ``offline.''
\end{subsection}
\end{section}

\begin{section}{Results}

\end{section}