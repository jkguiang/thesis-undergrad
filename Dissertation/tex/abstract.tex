%
%  Abstract
%

\begin{abstract}
\addcontentsline{toc}{chapter}{Abstract}
%todo: max 350 words

Progress in experimental High Energy physics can be derived from efforts in two fundamentally-linked domains: detector design and data analysis. Novel or otherwise improved detector designs push the boundaries of human perception, while rigorous, academically-motivated data analysis furthers the extent of human understanding. Therefore, both pursuits build towards the discovery of new physics, but they cannot be completely effective without the other. As such, this thesis covers them both. First, a simple, yet dynamic and effective simulation of the performance of a proposed addition to the CMS detector for the upcoming HL-LHC upgrade is described. Its programmatic construction allowed for clear and concise answers to challenging design questions posed during the upgrade's conceptual proposal and technical design. Second, a measurement of rare Higgs-boson decays, $H \rightarrow \rho+\gamma$ and $H \rightarrow \phi+ \gamma$, is performed using data based on a sample of proton-proton collisions collected by the Compact Muon Solenoid (CMS) detector at the Large Hadron Colider (LHC). Most notably, a Boosted Decision Tree (BDT) is used for distinguishing signal from background data after showing significant improvement over traditional, cut-based methods. This exploration is motivated by the possibility of measuring anomalous rates of these two particularly rare decay modes, which would reveal the existence of new physics.

%\abstractsignature
\end{abstract}


