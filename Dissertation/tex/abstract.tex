%
%  Abstract
%

\begin{abstract}
\addcontentsline{toc}{chapter}{Abstract}
%todo: max 350 words

Progress in experimental High Energy physics can be derived from efforts in two fundamentally-linked domains: detector design and data analysis. Novel or otherwise improved detector designs push the boundaries of human perception, while rigorous, academically-motivated data analysis furthers the extent of human understanding. Therefore, both pursuits build towards the discovery of new physics, but they cannot be completely effective without the other. As such, this thesis covers them both. First, a 3D model of the MIP Timing Detector (MTD) endcap and barrel timing layers -- a proposed addition to the Compact Muon Solenoid (CMS) for the High Luminocity LHC (HL-LHC) upgrade -- was designed using OpenSCAD, whose modular construction allowed for an entirely configurable model spanning several different designs. This model was used to simulate detector performance using particle trajectories from MinBias/$t\overline{t}$ CMS Monte Carlo (MC) samples propagated through a known map of the CMS magnetic field. Using this simulation, simple, yet accurate studies of the efficiency of various proposed designs were completed. For the latter, a search for the $H\rightarrow\rho\gamma$ and $H\rightarrow\phi\gamma$ decays was performed through associated $WH$ production using the full Run II 137 fb\textsuperscript{-1} dataset collected by CMS at the LHC. A Boosted Decision Tree (BDT) was trained on CMS background MC and a privately-generated signal sample and measured exclusion limits $\mathcal{B}(H\rightarrow\phi\gamma) \leq 4.1\times10^{-3}$ and $\mathcal{B}(H\rightarrow\rho\gamma) \leq 7.0\times10^{-3}$ to a 95\% confidence level.

%\abstractsignature
\end{abstract}


