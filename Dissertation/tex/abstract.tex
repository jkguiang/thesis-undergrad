%
%  Abstract
%

\begin{abstract}
\addcontentsline{toc}{chapter}{Abstract}
%todo: max 350 words

Progress in experimental High Energy physics can be derived from efforts in two fundamentally-linked domains: detector design and data analysis. As such, this thesis covers them both. First, a 3D model of the MIP Timing Detector (MTD) endcap and barrel timing layers -- a proposed addition to the Compact Muon Solenoid (CMS) for the High Luminocity LHC (HL-LHC) upgrade -- was designed using OpenSCAD, whose modular construction allowed for a configurable model spanning several different detector designs. Particle trajectories from MinBias/$t\overline{t}$ CMS Monte Carlo (MC) samples propagated through a known map of the CMS magnetic field in addition to the aforementioned 3D model were used to simulate detector performance for the purpose of completing simple, yet accurate studies -- primarily concerning the efficiency of various detector configurations -- conducted throughout the MTD's technical design process. Second, a search for the $H\rightarrow\rho\gamma$ and $H\rightarrow\phi\gamma$ decays was performed through associated $WH$ production. This search was motivated by the possibility of probing the Higgs boson couplings to light, flavored quarks, which could be enhanced by new physics. Using the full Run II 137 fb\textsuperscript{-1} $pp$ collision dataset collected by CMS at the LHC, no significant excess of events was observed above background, as expected from the Standard Model. For each analysis, a Boosted Decision Tree (BDT) was trained on CMS background MC and a privately-generated signal sample. Then, an optimal BDT discriminant was determined, and the branching ratio exclusion limits $\mathcal{B}(H\rightarrow\phi\gamma) \leq 4.1\times10^{-3}$ and $\mathcal{B}(H\rightarrow\rho\gamma) \leq 7.0\times10^{-3}$ were observed to a 95\% confidence level.

%\abstractsignature
\end{abstract}


