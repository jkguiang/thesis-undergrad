%
%  Abstract
%

\begin{abstract}
\addcontentsline{toc}{chapter}{Abstract}
%todo: max 350 words

Experimental High Energy physics can be divided into two fundamentally-linked domains: detector design and data analysis. Novel or otherwise improved detector designs push the boundaries of human perception, while rigorous, academically-motivated data analysis furthers the extend of human understanding. Therefore, both pursuits allow for the discovery of new physics, but they cannot be completely effective without the other, and it thus essential that a High Energy physicist is experienced in these two focuses. As such, this thesis covers them both. First, a simple, yet dynamic and effective simulation of the performance of a proposed addition to the CMS detector for the upcoming HL-LHC upgrade is described. Its programmatic construction allowed for clear and concise answers to challenging design questions posed during the upgrade's conceptual proposal and technical design. Second, a measurement of rare Higgs-boson decays, $H \rightarrow \rho , \gamma$ and $H \rightarrow \phi, \gamma$, is performed. This exploration is motivated by the possibility of finding anomalous rates of particularly rare decays which would reveal the existence of new physics.

%\abstractsignature
\end{abstract}


